{\begin{center}
\large\textbf{Non-technical Summary}
\end{center}
}


In the decade since the Great Recession, macroeconomics has made great progress by insisting that models be consistent with microeconomic evidence.  From the new generation of models, we take one specifically focused on reconciling apparent conflicts between micro and macro evidence about consumption dynamics and adapt it to incorporate two aspects of the coronavirus crisis. First, because the tidal wave of layoffs for employees of shuttered businesses will have a large impact on their income and spending, assumptions must be made about the employment dynamics of laid off workers.  Second, even consumers who remain employed will have restricted spending options.

On the first count, we model the likelihood that many of the people unemployed during the lockdown will be able to quickly return to their old jobs by assuming that the typical job loser has a two-thirds chance of being reemployed in the same or a similar job after each quarter of unemployment.  However, we expect that some kinds of jobs will not come back quickly after the lockdown, and that people who worked in these kinds of jobs will have more difficulty finding a new job.  We call these people the `deeply unemployed' and assume that there is a one-third chance each quarter that they become merely `normal unemployed.'  The `normal unemployed' have a jobfinding rate that matches average historical unemployment spell of 1.5 quarters.  Thus a deeply unemployed person expects to remain that way for three quarters on average, and then unemployed for another one and a half quarters.  When the pandemic hits, 10 percent of model households become normal unemployed and an additional 5 percent become deeply unemployed; in line with empirical evidence, the unemployment probabilities are skewed toward households who are young, unskilled and have low income.

On the second count, we model the restricted spending options by assuming that during the lockdown spending is less enjoyable.
Based on a tally of sectors that we judge to be substantially shuttered during the `lockdown,' we calibrate an $11$ percent reduction to spending.
Thus households will prefer to defer some of their consumption into the future, when it will yield them greater utility. 
In our primary scenario, we assume that this condition is removed with probability one-half after each quarter, so on average remains for two quarters.  When the `lockdown' ends, the buildup of savings by households who did not lose their jobs but whose spending was suppressed should result in a partial recovery in consumer spending, but in our primary scenario (without the CARES act), total consumer spending remains below its pre-crisis peak through the foreseeable future.

Our model captures the two primary features of the CARES Act that aim to bolster consumer spending:
\begin{enumerate}
\item The boost to unemployment insurance benefits, amounting to \$7,800 if unemployment lasts for 13 weeks.
\item The direct stimulus payments to households, amounting to \$1,200 per adult.
\end{enumerate}

We estimate that the combination of expanded unemployment insurance benefits and stimulus payments should be sufficient to allow a swift recovery in consumer spending to its pre-crisis levels under our primary scenario in which the lockdown ends after two quarters on average.
Overall, unemployment benefits account for about 30 percent of the total aggregate consumption response and stimulus payments explain the remainder.

Our analysis partitions households into three groups based on their employment state when the pandemic strikes and the lockdown begins.

First, households in our model who do not lose their jobs will initially build up their savings, both because of the lockdown-induced suppression of spending and because most of these households will receive a significant stimulus check, much of which the model says will be saved.
Even without the lockdown, we estimate that only about 20 percent of the stimulus money would be spent immediately upon receipt, consistent with evidence from prior stimulus packages about spending on nondurable goods and services.
Once the lockdown ends, the spending of the always-employed households rebounds strongly thanks to their healthy household finances.

The second category of households are the `normal unemployed,' job losers who perceive that it is likely they will be able to resume their old job (or get a similar new job) when the lockdown is over.
Our model predicts that the CARES Act will be particularly effective in stimulating their consumption, given the perception that their income shock will be largely transitory.  Our model predicts that by the end of 2021, the spending of this group will recover to the level it would have achieved in the absence of the pandemic (`baseline'); without the CARES Act, this recovery would take more than a year longer.

Finally, for households in the deeply unemployed category, our model says that the marginal propensity to consume (MPC) from the checks will be considerably smaller, because they know they must stretch that money for longer.
Even with the stimulus from the CARES Act, we predict that consumption spending for these households will not fully recover until the middle of 2023.
Even so, the act makes a big difference to their spending, particularly in the first six quarters after the crisis.
For both groups of unemployed households, the effect of the stimulus checks is dwarfed by the increased unemployment benefits, which arrive earlier and are much larger (per recipient).

\begin{comment}
Perhaps surprisingly, we find the effectiveness of the combined stimulus checks and unemployment benefits package for aggregate consumption is not substantially different from a package that distributed the same quantity of money equally between households.
The reason for this is twofold: first, the extra unemployment benefits in the CARES Act are generous enough that many of the `normally' unemployed remain financially sound and can afford to save a good portion of those benefits; second, the deeply unemployed expect their income to remain depressed for some time and therefore save more of the stimulus for the future.  In the model, the fact that they do \textit{not} spend immediately is actually a reflection of how desperately they anticipate these funds will be needed to make it through a long period of uncertainty.
While unemployment benefits do not strongly stimulate current consumption of the deeply unemployed, they do provide important disaster relief for those who may not be able to return to work for several quarters for an informal discussion).
\end{comment}

In addition to our primary scenario's relatively short lockdown period, we also consider a worse scenario in which the lockdown is expected to last for four quarters and the unemployment rate increases to 20 percent.
In this case, we find that the return of spending toward its no-pandemic path takes roughly three years. Moreover, the spending of deeply unemployed households will fall steeply unless the temporary unemployment benefits in the CARES Act are extended for the duration of the lockdown.







\newpage

